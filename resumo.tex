
Refatoração de código é uma prática importante no desenvolvimento de sistemas e um fator essencial para entender a evolução de um software.
Sendo assim, pesquisadores frequentemente reportam e discutem a prática de refatoração em sistemas reais.
Infelizmente, estudos empíricos sobre refatoração são frequentemente limitados pela dificuldade de se obter informações confiáveis sobre a atividade de refatoração e muitas questões permanecem em aberto.
Nesta tese, primeiro investigamos uma importante questão: \emph{por que desenvolvedores refatoram?}
Para esse fim, desenvolvemos dois estudos empíricos em larga escala, baseados na mineração de refatorações em históricos de versões.
Inicialmente, investigamos a relação entre a refatoração Extrair Método e reúso de código.
Após analizar mais de 10 mil revisões de 10 sistemas, encontramos evidências de que em 56,9\% dos casos tal refatoração é motivada pelo reúso de código.
Em seguida, investigamos as motivações para refatorações encontradas em sistemas de código aberto com base em respostas dos próprios desenvolvedores que as aplicaram.
Como resultado, compilamos um catálogo com 44 motivações distintas para 12 tipos de refatorações.
Tal catálogo revela que o esforço de refatoração é mais direcionado pela necessidade de evolução do sistema do que pela resolução de problemas de projeto conhecidos como \emph{code smells}.
Notadamente, Extrair Método é a refatoração mais versátil, servindo a 11 propósitos diferentes.
%Additionally, we found evidence that the IDE used by the developers affects the adoption of automated refactoring tools.
Em uma segunda linha de pesquisa, nós propomos RefDiff, uma nova ferramenta para mineração de refatorações em histórico de versões, com suporte a múltiplas linguagens de programação e alta precisão.
%As a second line of research, we propose RefDiff, a novel approach to mine refactorings from version histories that supports multiple programming languages and offers high precision and recall.
Nossa ferramenta introduz um algoritmo de detecção de refatorações baseado na \emph{Code Structure Tree} (CST)---uma representação do código fonte que abstrai as particularidades das linguagens de programação---e em uma métrica de similaridade de código baseada na técnica \emph{TF-IDF}.
Apesar do seu projeto multilinguagem, nossa avaliação revelou que nossa ferramenta tem precisão (96\%) e revocação (80\%) equivalentes ao estado da arte em ferramentas especializadas na linguagem Java.

\keywords{Refatoração, Evolução de Software, Mineração de Repositórios}